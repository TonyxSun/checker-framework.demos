% -*- Mode: LaTeX -*-
\documentclass{llncs} % 10pt if not specified
\usepackage{pslatex}
\usepackage{relsize}
\usepackage{url}

% The proposal should include:
% Title.
% Authors names.
% Abstract (up to 300 words).
% Target audience.
% Expected added value.
% Needed Material: please describe the material that the organization is expected to provide. 
% Name of the events in which the tutorial has already been presented (if any).
% A short bio of the authors.

% The proposals should be sent in a PDF, until December 15, 2014, by email to:
% lanto@lifia.info.unlp.edu.ar and ernesto@lcc.uma.es.

\begin{document}

\title{Lightweight software verification \\ with pluggable type-checking}
\author{Michael D. Ernst}
\institute{University of Washington \quad and \quad Universidad
    de Buenos Aires \\ \email{mernst@cs.washington.edu} \\ \url{http://checkerframework.org/}}
\maketitle

\begin{abstract} % up to 300 words

Software developers often rely on run-time exceptions to indicate bugs
in their code.  It would be better to use verification to prove the absence
of bugs, but verification tends to be difficult to use.

We propose a lightweight software verification approach, called \emph{pluggable
type-checking}, that is easy to use, extensible, and provides a compile-time
guarantee that certain bugs are not present in the code.  Pluggable
type-checking permits a software developer to refine the built-in type
system of a programming language to catch additional errors, such as null
pointer dereferences or race conditions.

This approach has been implemented for Java and is available in an
open-source tool, the Checker Framework
(\url{http://checkerframework.org/}).  Oracle Corporation is so excited
about this technology that Java 8 contains syntactic support for pluggable
types.

This verification approach is relevant to multiple constituencies.

\textbf{Researchers} can build upon the framework to quickly create new
program analysis tools.  Previously, evaluating a new type system required
building a compiler.  Now, is it easier to experimentally evaluate a new
type system, because the type-checker implementation is only a few 
lines of code.

\textbf{Educators} can introduce software verification in a practical
context, enabling students to learn by doing and bringing theory to life.
The Checker Framework has been successfully used in
the first or second programming class for computer science majors, and also
in more advanced classes.

\textbf{Practitioners} can use pluggable type-checking to find bugs or to
prove the absence of bugs.  Use of pluggable type-checking improves code
quality and design, and the types act as machine-checked documentation.
The Checker Framework is in daily use at corporations such as Google.


Attendees will leave the tutorial with a greater appreciation of the theory and
practice of pluggable type-checking.  They will be prepared to use it
for research, teaching, or software development.


% \keywords{computational geometry, graph theory, Hamilton cycles}
\end{abstract}

\section{Target audience}

As described in the abstract, this talk is relevant to researchers,
educators, and practitioners.

To obtain the greatest benefit from the talk, an audience member should be
familiar with Java and its type system.  The audience member should know
the basics of object-oriented typing, such as the type hierarchy (also
called the class hierarchy, which is induced by subtyping or subclassing)
and the fact that an assignment is legal iff the declared type of the
left-hand-side is a supertype of the right-hand-side.  The audience does
not need to know details of the Java language such as the rather obscure
semantics of wildcards.  Nor does the audience need expertise with type
systems (such as different varieties of polymorphism and their semantics);
the tutorial will present any the needed background.

\section{Expected added value}

A little-known and under-appreciated feature of Java 8 is type annotations, an
innovative feature found in no other mainstream language.  Type annotations
ease the documentation task by allowing a programmer to write short,
precise annotations instead of natural-language Javadoc explanations.  More
importantly, they enable a tool to verify that the documentation is correct
and up to date, and that the code is free of certain bugs.

This interactive tutorial will explain how and why to use Java 8's type
annotations.  After a brief introduction, the audience will dive into using
them.  We will start with tutorials of using existing type-checkers on
example code.  Afterward, the audience will have two options regarding how
to proceed:  creating new type-checkers, or applying existing type-checkers
to real-world code.  By getting practice, the audience will overcome any
reluctance regarding the difficulty of using either verification tools or
type-checkers (both of which have gotten a bad reputation).

During a brainstorming session, the audience will come to better understand
the theory and practice of type-checking, including what sorts of problems
it is good for (which is far more than most people realize) and what sorts
of problems it is not appropriate for.

The audience will leave the session prepared to use pluggable type-checking
for research, teaching, or software development.


\section{Needed material}

% Needed Material: please describe the material that the organization is expected to provide. 

\begin{enumerate}
\item
A data projector for displaying the presenter's laptop computer --- so the
audience can view slides and demos.
\item
Power and Internet connectivity for the participants --- so the audience
can participate in exercises on their laptop computers.
\item
A whiteboard or a flip chart (large sheet of paper), plus markers --- for
writing down ideas while brainstorming.
\end{enumerate}


\section{Previous presentations}

I and my colleagues have given a number of presentations on the Checker
Framework, including at the industry conferences JavaOne (2009, 2012, 2013,
2014), OSBridge (2011), and OSCON (2012), and at the academic venues such
as OOPSLA (2009) and TAPAS (2014).  So, part of this tutorial draws on rich
material that has been well-vetted.  Part of the material is new, however.
This includes the part about using the Checker Framework in teaching, which
has never been presented before but seems appropriate for the largely
academic audience at CibSE\@.


\section{Author bio}

Michael D. Ernst is a Professor in the Computer Science \& Engineering
department at the University of Washington.

Ernst's research aims to make software more reliable, more secure, and
easier (and more fun!) to produce. His primary technical interests are in
software engineering, programming languages, type theory, security, program
analysis, bug prediction, testing, and verification. Ernst's research
combines strong theoretical foundations with realistic experimentation,
with an eye to changing the way that software developers work.

Ernst's awards include the inaugural John Backus Award (2009) and the NSF
CAREER Award (2002). His research has received an ACM SIGSOFT Impact Paper
Award (2013), 7 ACM Distinguished Paper Awards (ISSTA 2014, ESEC/FSE 2011,
ISSTA 2009, ESEC/FSE 2007, ICSE 2007, ICSE 2004, ESEC/FSE 2003), an ECOOP
2011 Best Paper Award, honorable mention in the 2000 ACM doctoral
dissertation competition, and other honors. In 2013, Microsoft Academic
Search ranked Ernst \#2 in the world, in software engineering research
contributions over the past 10 years.  He was honored as a Java Rock Star
for his well-received presentations at the JavaOne conference, including
previous versions of this tutorial.

Dr. Ernst was previously a tenured professor at MIT, and before that a researcher at Microsoft Research.

More information is available at his homepage: \url{http://homes.cs.washington.edu/~mernst/}.

\end{document}

%%  LocalWords:  Universidad de wildcards OSBridge TAPAS bio
